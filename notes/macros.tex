%%%%%%%%%%%% Custom symbols %%%%%%%%%%%%%

\newsavebox{\circlehorlinebox}
\newcommand{\circlehorline}[0]{%
\begin{tikzpicture}[scale=0.20]%
\draw (0,0.5cm) -- (1.6cm,0.5cm);%
\draw (0.8cm,0.5cm) circle(0.4cm);
\end{tikzpicture}%
}
\savebox{\circlehorlinebox}{\circlehorline}
\newcommand*\modusponens{\,\vcenter{\hbox{\usebox{\circlehorlinebox}}}\,}

%%%%%%%%%%%%%%%%%%%%%%%%%%%%%%%%%%%%%%%%%

%% Comments
\newcommand*{\klara}[1]{\textcolor{orange}{#1}}

%% Contexts
\newcommand*{\hatmarker}[1]{{}_{\blacktriangleright\hat{#1}}}
\newcommand*{\tentry}[2]{#1\ast #2}
\newcommand*{\eentry}[2]{#1 : #2}

%% Well-formedness
\newcommand*{\synthmode}[4]{#1\vdash #2\,\textcolor{red}{\Rightarrow}\,{#3}\leadsto\elab{#4}}
\newcommand*{\checkmode}[4]{#1\vdash #2\,\textcolor{blue}{\Leftarrow}\,{#3}\leadsto\elab{#4}}
\newcommand*{\synthspine}[5]{#1\vdash #2\bullet #3\,\textcolor{violet}{\rightrightarrows}\,{#4}\leadsto\elab{#5}}
\newcommand*{\algsynthmode}[5]{#1\vdash #2\,\textcolor{red}{\Rightarrow}\,{#3}\leadsto\elab{#4}\dashv #5}
\newcommand*{\algcheckmode}[5]{#1\vdash #2\,\textcolor{blue}{\Leftarrow}\,{#3}\leadsto\elab{#4}\dashv #5}
\newcommand*{\algsynthspine}[6]{#1\vdash #2\bullet #3\,\textcolor{violet}{\rightrightarrows}\,{#4}\leadsto\elab{#5}\dashv #6}
\newcommand*{\wfT}[2]{#1\vdash #2}
\newcommand*{\wfC}[4]{#1\vdash #2 : #3 \triangleright #4}
\newcommand*{\wfContext}[1]{\vdash #1}

%% Evaluation

%% Disjointness
\newcommand*{\starr}[2]{\tentry{#1}{#2}}
\newcommand*{\starax}[2]{#1\ast_{ax} #2}
\newcommand*{\disjoint}[3]{#1 \vdash \starr{#2}{#3}}
\newcommand*{\algdisjoint}[4]{#1 \vdash \starr{#2}{#3}\,\leadsto #4}
\newcommand*{\udisjoint}[2]{{#1}{\vdash_d}{#2}}
\newcommand*{\algUdisjoint}[3]{#1\vdash_d #2\,\leadsto #3}

%% Declarative subtyping
\newcommand*{\subtypeof}{\prec :}
\newcommand*{\subt}[2]{#1\subtypeof #2}
\newcommand*{\subtype}[3]{\subt{#1}{#2}\,\elab{\leadsto #3}}
\newcommand*{\Subtype}[4]{#1\vdash\subtype{#2}{#3}{#4}}

%% Algorithmic subtyping
\newcommand*{\algsubtypeof}{<:}
\newcommand*{\algsubt}[2]{#1\algsubtypeof #2}
\newcommand*{\algsubtype}[3]{\algsubt{#1}{#2}\leadsto\elab{#3}}
\newcommand*{\algSubMain}[4]{\algsubtype{#1}{#2}{#3};#4}
\newcommand*{\algSubRight}[6]{#1;#2\vdash_R\algSubMain{#3}{#4}{#5}{#6}}
\newcommand*{\algSubLeft}[9]{#1;#2;#3;#4;\elab{#5}\vdash_L\algSubMain{#6}{#7}{#8}{#9}}

%% Unification
\newcommand*{\unify}[4]{\mathsf{unify}(#1;\,#2;\,#3)\,\leadsto #4}
\newcommand*{\unifyB}[4]{\mathsf{unify_B}(#1;\,#2;\,#3)\,\leadsto #4}
\newcommand*{\unifyM}[4]{\mathsf{unify_M}(#1;\,#2;\,#3)\,\leadsto #4}

%% %% Auxiliary functions
\newcommand*{\distarrowqueue}[3]{\llbracket #1 \rrbracket_{\&}^{#2,#3}}
\newcommand*{\distarrowqueueC}[4]{\llbracket #1 \rrbracket_{\&}^{#2;#3,#4}}
\newcommand*{\toparrowqueue}[1]{\llbracket #1 \rrbracket_{\top}}
\newcommand*{\arrowqueue}[2]{#1 \twoheadrightarrow #2}

%% Expressions
\newcommand*{\True}{\mathsf{true}}
\newcommand*{\False}{\mathsf{false}}
\newcommand*{\tabs}[3]{\Lambda(\tentry{#1}{#2}).#3}
\newcommand*{\tabst}[2]{\Lambda #1.#2}
\newcommand*{\abs}[2]{\lambda #1.#2}
\newcommand*{\record}[2]{\{#1\,=\,#2\}}

%% Types
\newcommand*{\nat}{\mathsf{Nat}}
\newcommand*{\bool}{\mathsf{Bool}}
\newcommand*{\Tabs}[3]{\forall(\tentry{#1}{#2}).#3}
\newcommand*{\Tabst}[2]{\forall #1.#2}
\newcommand*{\trecord}[2]{\{#1\,:\,#2\}}

%% Substitution
\newcommand*{\substitution}[2]{#1\mapsto #2}
\newcommand*{\unification}[2]{\substitution{\hat{#1}}{#2}}
\newcommand*{\wfSubst}[3]{#1\vdash #2\,\leadsto #3}

%% Translation
\newcommand*{\toTarget}[1]{|#1|}
\newcommand*{\elab}[1]{\colorbox{lightgray}{$#1$}}

%% Coercions
\newcommand*{\idC}[1]{\mathsf{id}_{#1}}
\newcommand*{\compC}[2]{#1\,\circ\, #2}
\newcommand*{\topC}[1]{\mathsf{top}_{#1}}
\newcommand*{\topArrC}{\mathsf{top}_{\mathsf\to}}
\newcommand*{\topRecC}[1]{\mathsf{top}_{\{ #1 \}}}
\newcommand*{\topAllC}{\mathsf{top}_{\mathsf\forall}}
\newcommand*{\pair}[2]{\langle #1, #2\rangle}
\newcommand*{\pairC}[2]{\pair{#1}{#2}}
\newcommand*{\projlC}[2]{\mathsf{pr}_{\mathsf{1}}^{#1,#2}}
\newcommand*{\projrC}[2]{\mathsf{pr}_{\mathsf{2}}^{#1,#2}}
\newcommand*{\arrC}[2]{#1\to #2}
\newcommand*{\distArrC}[3]{\mathsf{dist}_{\to}^{#1,#2,#3}}
\newcommand*{\distRecC}[3]{\mathsf{dist}_{\{ #1 \}}^{#2,#3}}
\newcommand*{\distAllC}[3]{\mathsf{dist}_{\forall}^{#1,#2,#3}}
\newcommand*{\alllC}[2]{#1 @ #2}
\newcommand*{\allrC}[2]{\Lambda #1.#2}
\newcommand*{\mpC}[2]{#1\modusponens #2}

%% Coercion contexts
\newcommand*{\arrCC}[2]{{}^{#2\to}#1}
\newcommand*{\projlCC}[3]{{#1}^{#2,#3}_{\circ\mathsf{pr_1}}}
\newcommand*{\projrCC}[3]{{#1}^{#2,#3}_{\circ\mathsf{pr_2}}}
\newcommand*{\mpCC}[5]{{#1}_{#2;#3}^{#4,#5}}
\newcommand*{\recCC}[2]{{#1}_{\{#2\}}}
\newcommand*{\alllCC}[2]{{#1}^{@ #2}}

%% Auxialiary premises / Phrases
\newcommand*{\fresh}[1]{#1~\text{fresh}}
\newcommand*{\notarrow}[1]{#1\not\to} %~\text{not arrow}}
\newcommand*{\qbrack}[1]{\llbracket #1\rrbracket}
\newcommand*{\dblarrow}[2]{#1\twoheadrightarrow #2}
